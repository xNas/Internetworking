\documentclass[12pt,a4paper,fleqn]{article}
	\usepackage[spanish,activeacute]{babel}

	\title{Internetworking - Cliente TFTP (RFC 1350)}
	\author{
		Cristian Herrera\\{\small Internetworking UTN -FRLP}\\{\small \texttt{cristianalfredoherrera@gmail.com}}\\
		Nahuel Santos\\{\small Internetworking UTN -FRLP}\\{\small \texttt{nahuelsantos@gmail.com}}\\
	}
	\begin{document}
		\maketitle

		\section{Propuesta}
			\subsection{Motivaci\'on}
				La motivaci\'on para realizar el siguiente trabajo es la de poder contar con una aplicaci\'on cliente - servidor que permita la transferencia de archivos de una forma sencilla, sin la necesidad de tener que configurar un servicio para tal fin.
			\subsection{Cliente TFTP (seg\'un RFC 1350)}
				El trabajo consistir\'a en la programaci\'on de un cliente que cumpla con el protocolo TFTP\footnote{Trivial File Transfer Protocol, es un protocolo est\'andar para la transferencia de archivos que utiliza la capa de transporte UDP} de acuerdo a la RFC 1350\footnote{http://www.faqs.org/rfcs/rfc1350.html}.\\\\
				El software deber\'a ser capaz de conectarse a cualquier servidor TFTP.

\end{document}
